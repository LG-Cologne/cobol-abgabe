\chapter{Testdokumentation}\label{ch:testdokumentation}
Alle Testfälle können wie beschrieben in~\enquote{\nameref{sec:testen-der-beispiele}}~ausgeführt werden.
Für eine klare Struktur wurden Sie in 5 verschiedene Testgruppen eingeteilt:
\begin{enumerate}[label={\textbf{Gruppe~\arabic*:}}, ref={Gruppe~\arabic*}, leftmargin=*, noitemsep]
    \label{enm:tesgruppen}
    \item Tests, welche aus der Aufgabenstellung hervorgehen.
    \item Überprüfungen der Eingabe im T9- bzw. Normalmodus.
    \item Überprüfungen der Eingabe im Explizitmodus.
    \item Tests, welche speziell auf die Semantik im Explizitmodus eingehen.
    \item Tests, welche die Reihenfolge der Wortvorschläge überprüfen
    \item Tests, welche das Einlesen und Ausgeben von Wörterbüchern überprüfen.
    \item Tests, zum Überprüfen der beschriebenen Sonderfälle (Kapitel~\ref{subsec:sonderfaelle})
\end{enumerate}

\definecolor{fhfarbe}{HTML}{00605E}


\section{Definierte Tests}\label{sec:definierte-tests}
Die in der Aufgabenstellung definierten Tests werden im Folgenden ausführlich beschrieben.
Dabei werden die allgemeine Funktionalität des Programms getestet und Normalfälle von vielen Programmkomponenten abgedeckt.
\subsection*{Normalfälle}\label{subsec:def-normalfaelle}

\paragraph*{T1.1}
Beginnend mit einem Leeren Wörterbuch wird getestet, ob der Satz \glqq DER HUT IST EIN FES.\grqq{} gebildet werden kann.
Wörterbuch:
\begin{lstlisting}[escapechar=\%]
Moechten Sie ein externes Woerterbuch einlesen? ja (*) / nein (#)
%\textcolor{fhfarbe}{\#}%
Kein Woerterbuch geladen.
Beginnen Sie einen neuen Satz oder beenden Sie das Programm mit (0).
Bitte geben Sie ein kodiertes Wort ein:
%\textcolor{fhfarbe}{3371}%
Im Woerterbuch wurde keine passender Eintrag gefunden!
Bitte Wort im Explizit-Modus eingeben:
%\textcolor{fhfarbe}{3132731}%
DER
Wort in Ordnung? ja (*) / nein (#)
%\textcolor{fhfarbe}{*}%
Das Wort DER wurde mit Code 337 im Woerterbuch abgespeichert!
Bitte geben Sie ein kodiertes Wort ein:
%\textcolor{fhfarbe}{4881}%
Im Woerterbuch wurde keine passender Eintrag gefunden!
Bitte Wort im Explizit-Modus eingeben:
%\textcolor{fhfarbe}{4282811}%
HUT
Wort in Ordnung? ja (*) / nein (#)
%\textcolor{fhfarbe}{*}%
Das Wort HUT wurde mit Code 488 im Woerterbuch abgespeichert!
Bitte geben Sie ein kodiertes Wort ein:
%\textcolor{fhfarbe}{4781}%
Im Woerterbuch wurde keine passender Eintrag gefunden!
Bitte Wort im Explizit-Modus eingeben:
%\textcolor{fhfarbe}{4374811}%
IST
Wort in Ordnung? ja (*) / nein (#)
%\textcolor{fhfarbe}{*}%
Das Wort IST wurde mit Code 478 im Woerterbuch abgespeichert!
Bitte geben Sie ein kodiertes Wort ein:
%\textcolor{fhfarbe}{3461}%
Im Woerterbuch wurde keine passender Eintrag gefunden!
Bitte Wort im Explizit-Modus eingeben:
%\textcolor{fhfarbe}{3243621}%
EIN
Wort in Ordnung? ja (*) / nein (#)
%\textcolor{fhfarbe}{*}%
Das Wort EIN wurde mit Code 346 im Woerterbuch abgespeichert!
Bitte geben Sie ein kodiertes Wort ein:
%\textcolor{fhfarbe}{3370}%
DER
Wort in Ordnung? ja (*) / nein (#)
%\textcolor{fhfarbe}{\#}%
Bitte Wort im Explizit-Modus eingeben:
%\textcolor{fhfarbe}{3332740}%
FES
Wort in Ordnung? ja (*) / nein (#)
%\textcolor{fhfarbe}{*}%
Das Wort FES wurde mit Code 337 im Woerterbuch abgespeichert!
Eingegebener Satz:
DER HUT IST EIN FES.
Beginnen Sie einen neuen Satz oder beenden Sie das Programm mit (0).
Bitte geben Sie ein kodiertes Wort ein:
%\textcolor{fhfarbe}{0}%
Programm Ende.
\end{lstlisting}

\begin{center}
    \fbox{\begin{minipage}{15em}
              337 DER 1 \\
              346 EIN 1 \\
              337 FES 1 \\
              488 HUT 1 \\
              478 IST 1
    \end{minipage}}
    \captionof{figure}{Wörterbuch nach Test 1.1.}
    \label{fig:wbuch1}
\end{center}
\paragraph*{T1.2} Es sollen die beiden Sätze \glqq DER SATZ IST KURZ. EIN FES IST EIN HUT.\grqq{} konstruiert werden.
Dabei wird das Wörterbuch~\ref{fig:wbuch1} aus dem vorherigen Test verwendet.
\begin{lstlisting}[escapechar=\%]
Moechten Sie ein externes Woerterbuch einlesen? ja (*) / nein (#)
%\textcolor{fhfarbe}{*}%
Bitte geben sie den Dateinamen des Woerterbuchs ein.
%\textcolor{fhfarbe}{Woerterbuch-out.txt}%
Woerterbuch erfolgreich eingelesen.
Beginnen Sie einen neuen Satz oder beenden Sie das Programm mit (0).
Bitte geben Sie ein kodiertes Wort ein:
%\textcolor{fhfarbe}{3371}%
DER
Wort in Ordnung? ja (*) / nein (#)
%\textcolor{fhfarbe}{*}%
Bitte geben Sie ein kodiertes Wort ein:
%\textcolor{fhfarbe}{72891}%
Im Woerterbuch wurde keine passender Eintrag gefunden!
Bitte Wort im Explizit-Modus eingeben:
%\textcolor{fhfarbe}{742181941}%
SATZ
Wort in Ordnung? ja (*) / nein (#)
%\textcolor{fhfarbe}{*}%
Das Wort SATZ wurde mit Code 7289 im Woerterbuch abgespeichert!
Bitte geben Sie ein kodiertes Wort ein:
%\textcolor{fhfarbe}{4781}%
IST
Wort in Ordnung? ja (*) / nein (#)
%\textcolor{fhfarbe}{*}%
Bitte geben Sie ein kodiertes Wort ein:
%\textcolor{fhfarbe}{58790}%
Im Woerterbuch wurde keine passender Eintrag gefunden!
Bitte Wort im Explizit-Modus eingeben:
%\textcolor{fhfarbe}{528273940}%
KURZ
Wort in Ordnung? ja (*) / nein (#)
%\textcolor{fhfarbe}{*}%
Das Wort KURZ wurde mit Code 5879 im Woerterbuch abgespeichert!
Eingegebener Satz:
DER SATZ IST KURZ.
Beginnen Sie einen neuen Satz oder beenden Sie das Programm mit (0).
Bitte geben Sie ein kodiertes Wort ein:
%\textcolor{fhfarbe}{3461}%
EIN
Wort in Ordnung? ja (*) / nein (#)
%\textcolor{fhfarbe}{*}%
Bitte geben Sie ein kodiertes Wort ein:
%\textcolor{fhfarbe}{3371}%
DER
Wort in Ordnung? ja (*) / nein (#)
%\textcolor{fhfarbe}{\#}%
FES
Wort in Ordnung? ja (*) / nein (#)
%\textcolor{fhfarbe}{*}%
Bitte geben Sie ein kodiertes Wort ein:
%\textcolor{fhfarbe}{4781}%
IST
Wort in Ordnung? ja (*) / nein (#)
%\textcolor{fhfarbe}{*}%
Bitte geben Sie ein kodiertes Wort ein:
%\textcolor{fhfarbe}{3461}%
EIN
Wort in Ordnung? ja (*) / nein (#)
%\textcolor{fhfarbe}{*}%
Bitte geben Sie ein kodiertes Wort ein:
%\textcolor{fhfarbe}{4880}%
HUT
Wort in Ordnung? ja (*) / nein (#)
%\textcolor{fhfarbe}{*}%
Eingegebener Satz:
EIN FES IST EIN HUT.
Beginnen Sie einen neuen Satz oder beenden Sie das Programm mit (0).
Bitte geben Sie ein kodiertes Wort ein:
%\textcolor{fhfarbe}{0}%
Programm Ende.
\end{lstlisting}
\begin{center}
    \fbox{\begin{minipage}{15em}
              337 DER 2 \\
              346 EIN 3\\
              337 FES 2\\
              488 HUT 2\\
              478 IST 3\\
              5879 KURZ 1\\
              7289 SATZ 1
    \end{minipage}}
    \captionof{figure}{Wörterbuch nach Test 1.2.}
    \label{fig:wbuch2}
\end{center}


\section{Normalmodus-Tests}\label{sec:normalmodus-tests}
In den folgenden Testfällen wird geprüft, ob das Programm bei Verletzung der in Kapitel~\ref{subsec:syntaktische-fehler} syntaktischen Regeln für den normalen Modus passend reagiert.
In allen Fällen sollte ein Fehler gemeldet und eine neue Eingabe gefordert werden.

\subsection*{Fehlerfälle}\label{subsec:normalmodus-fehlerfaelle}

\paragraph*{T2.1} Leerzeichen im Wort.
\begin{lstlisting}[escapechar=\%]
Bitte geben Sie ein kodiertes Wort ein:
%\textcolor{fhfarbe}{76 3892730}%
Leerzeichen innerhalb des Wortes sind nicht erlaubt!
Bitte geben Sie ein kodiertes Wort ein:
\end{lstlisting}

\paragraph*{T2.2} Ungültiges Wortende.
\begin{lstlisting}[escapechar=\%]
Bitte geben Sie ein kodiertes Wort ein:
%\textcolor{fhfarbe}{76389273}%
Es muss genau ein Leerzeichen (1) oder Punkt (0) am Wortende stehen!
Bitte geben Sie ein kodiertes Wort ein:
\end{lstlisting}

\paragraph*{T2.3} Unerlaubte Zeichen-Eingabe.
\begin{lstlisting}[escapechar=\%]
Bitte geben Sie ein kodiertes Wort ein:
%\textcolor{fhfarbe}{76389g2a730}%
Die Eingabe darf nur Ziffern von 0-9 enthalten!
Bitte geben Sie ein kodiertes Wort ein:
\end{lstlisting}

\paragraph*{T2.4} Zeichen nach Wortende.
\begin{lstlisting}[escapechar=\%]
Bitte geben Sie ein kodiertes Wort ein:
%\textcolor{fhfarbe}{763890273}%
Nach dem Leerzeichen bzw. Punkt duerfen keine weiteren Zeichen folgen!
Bitte geben Sie ein kodiertes Wort ein:
\end{lstlisting}

\paragraph*{T2.5} Mehr als ein Wortende.
\begin{lstlisting}[escapechar=\%]
Bitte geben Sie ein kodiertes Wort ein:
%\textcolor{fhfarbe}{7638927310}%
Es muss genau ein Leerzeichen (1) oder Punkt (0) am Wortende stehen!
Bitte geben Sie ein kodiertes Wort ein:
\end{lstlisting}

\paragraph*{T2.6} Leerzeichen im Wort und ungültiges Ende.
\begin{lstlisting}[escapechar=\%]
Bitte geben Sie ein kodiertes Wort ein:
%\textcolor{fhfarbe}{76 389273}%
Leerzeichen innerhalb des Wortes sind nicht erlaubt!
Es muss genau ein Leerzeichen (1) oder Punkt (0) am Wortende stehen!
Bitte geben Sie ein kodiertes Wort ein:
\end{lstlisting}

\paragraph*{T2.7} Leerzeichen im Wort und unerlaubte Zeichen.
\begin{lstlisting}[escapechar=\%]
Bitte geben Sie ein kodiertes Wort ein:
%\textcolor{fhfarbe}{763 89g2a730}%
Die Eingabe darf nur Ziffern von 0-9 enthalten!
Leerzeichen innerhalb des Wortes sind nicht erlaubt!
Bitte geben Sie ein kodiertes Wort ein:
\end{lstlisting}

\paragraph*{T2.8} Leerzeichen im Wort und Zeichen nach Wortende.
\begin{lstlisting}[escapechar=\%]
Bitte geben Sie ein kodiertes Wort ein:
%\textcolor{fhfarbe}{7638902 73}%
Leerzeichen innerhalb des Wortes sind nicht erlaubt!
Nach dem Leerzeichen bzw. Punkt duerfen keine weiteren Zeichen folgen!
Bitte geben Sie ein kodiertes Wort ein:
\end{lstlisting}

\paragraph*{T2.9} Leerzeichen im Wort und mehrfaches Wortende.
\begin{lstlisting}[escapechar=\%]
Bitte geben Sie ein kodiertes Wort ein:
%\textcolor{fhfarbe}{76389 27310}%
Leerzeichen innerhalb des Wortes sind nicht erlaubt!
Es muss genau ein Leerzeichen (1) oder Punkt (0) am Wortende stehen!
Bitte geben Sie ein kodiertes Wort ein:
\end{lstlisting}

\paragraph*{T2.10} Unerlaubte Zeichen und fehlendes Wortende.
\begin{lstlisting}[escapechar=\%]
Bitte geben Sie ein kodiertes Wort ein:
%\textcolor{fhfarbe}{76389g2a73}%
Die Eingabe darf nur Ziffern von 0-9 enthalten!
Es muss genau ein Leerzeichen (1) oder Punkt (0) am Wortende stehen!
Bitte geben Sie ein kodiertes Wort ein:
\end{lstlisting}

\paragraph*{T2.11} Unerlaubte Zeichen und Zeichen nach Wortende.
\begin{lstlisting}[escapechar=\%]
Bitte geben Sie ein kodiertes Wort ein:
%\textcolor{fhfarbe}{76389g2a703}%
Die Eingabe darf nur Ziffern von 0-9 enthalten!
Nach dem Leerzeichen bzw. Punkt duerfen keine weiteren Zeichen folgen!
Bitte geben Sie ein kodiertes Wort ein:
\end{lstlisting}

\paragraph*{T2.12} Unerlaubte Zeichen und mehrfaches Wortende.
\begin{lstlisting}[escapechar=\%]
Bitte geben Sie ein kodiertes Wort ein:
%\textcolor{fhfarbe}{76389g2a7310}%
Die Eingabe darf nur Ziffern von 0-9 enthalten!
Es muss genau ein Leerzeichen (1) oder Punkt (0) am Wortende stehen!
Bitte geben Sie ein kodiertes Wort ein:
\end{lstlisting}

\paragraph*{T2.13} Mehrfaches Wortende und Zeichen nach Wortende.
\begin{lstlisting}[escapechar=\%]
Bitte geben Sie ein kodiertes Wort ein:
%\textcolor{fhfarbe}{76389027310}%
Es muss genau ein Leerzeichen (1) oder Punkt (0) am Wortende stehen!
Bitte geben Sie ein kodiertes Wort ein:
\end{lstlisting}

\paragraph*{T2.14} Leerzeichen im Wort und unerlaubte Zeichen und fehlendes Wortende.
\begin{lstlisting}[escapechar=\%]
Bitte geben Sie ein kodiertes Wort ein:
%\textcolor{fhfarbe}{76389g2 a73}%
Die Eingabe darf nur Ziffern von 0-9 enthalten!
Leerzeichen innerhalb des Wortes sind nicht erlaubt!
Es muss genau ein Leerzeichen (1) oder Punkt (0) am Wortende stehen!
Bitte geben Sie ein kodiertes Wort ein:
\end{lstlisting}

\paragraph*{T2.15} Alle vorherigen Fehler nacheinander und anschliessend korrekte Eingabe.
\begin{lstlisting}[escapechar=\%]
Bitte geben Sie ein kodiertes Wort ein:
%\textcolor{fhfarbe}{76 3892730}%
Leerzeichen innerhalb des Wortes sind nicht erlaubt!
Bitte geben Sie ein kodiertes Wort ein:
%\textcolor{fhfarbe}{76389273}%
Es muss genau ein Leerzeichen (1) oder Punkt (0) am Wortende stehen!
Bitte geben Sie ein kodiertes Wort ein:
%\textcolor{fhfarbe}{76389g2a730}%
Die Eingabe darf nur Ziffern von 0-9 enthalten!
Bitte geben Sie ein kodiertes Wort ein:
%\textcolor{fhfarbe}{763890273}%
Nach dem Leerzeichen bzw. Punkt duerfen keine weiteren Zeichen folgen!
Bitte geben Sie ein kodiertes Wort ein:
%\textcolor{fhfarbe}{7638927310}%
Es muss genau ein Leerzeichen (1) oder Punkt (0) am Wortende stehen!
Bitte geben Sie ein kodiertes Wort ein:
%\textcolor{fhfarbe}{76 389273}%
Leerzeichen innerhalb des Wortes sind nicht erlaubt!
Es muss genau ein Leerzeichen (1) oder Punkt (0) am Wortende stehen!
Bitte geben Sie ein kodiertes Wort ein:
%\textcolor{fhfarbe}{763 89g2a730}%
Die Eingabe darf nur Ziffern von 0-9 enthalten!
Leerzeichen innerhalb des Wortes sind nicht erlaubt!
Bitte geben Sie ein kodiertes Wort ein:
%\textcolor{fhfarbe}{7638902 73}%
Leerzeichen innerhalb des Wortes sind nicht erlaubt!
Nach dem Leerzeichen bzw. Punkt duerfen keine weiteren Zeichen folgen!
Bitte geben Sie ein kodiertes Wort ein:
%\textcolor{fhfarbe}{76389 27310}%
Leerzeichen innerhalb des Wortes sind nicht erlaubt!
Es muss genau ein Leerzeichen (1) oder Punkt (0) am Wortende stehen!
Bitte geben Sie ein kodiertes Wort ein:
%\textcolor{fhfarbe}{76389g2a73}%
Die Eingabe darf nur Ziffern von 0-9 enthalten!
Es muss genau ein Leerzeichen (1) oder Punkt (0) am Wortende stehen!
Bitte geben Sie ein kodiertes Wort ein:
%\textcolor{fhfarbe}{76389g2a703}%
Die Eingabe darf nur Ziffern von 0-9 enthalten!
Nach dem Leerzeichen bzw. Punkt duerfen keine weiteren Zeichen folgen!
Bitte geben Sie ein kodiertes Wort ein:
%\textcolor{fhfarbe}{76389g2a7310}%
Die Eingabe darf nur Ziffern von 0-9 enthalten!
Es muss genau ein Leerzeichen (1) oder Punkt (0) am Wortende stehen!
Bitte geben Sie ein kodiertes Wort ein:
%\textcolor{fhfarbe}{76389027310}%
Es muss genau ein Leerzeichen (1) oder Punkt (0) am Wortende stehen!
Bitte geben Sie ein kodiertes Wort ein:
%\textcolor{fhfarbe}{76389g2 a73}%
Die Eingabe darf nur Ziffern von 0-9 enthalten!
Leerzeichen innerhalb des Wortes sind nicht erlaubt!
Es muss genau ein Leerzeichen (1) oder Punkt (0) am Wortende stehen!
Bitte geben Sie ein kodiertes Wort ein:
%\textcolor{fhfarbe}{763892730}%
Im Woerterbuch wurde keine passender Eintrag gefunden!
Bitte Wort im Explizit-Modus eingeben:
%\textcolor{fhfarbe}{74633381912173320}%
 SOFTWARE
Wort in Ordnung? ja (*) / nein (#)
%\textcolor{fhfarbe}{*}%
Das Wort SOFTWARE wurde mit Code 76389273 im Woerterbuch abgespeichert!
Eingegebener Satz:
 SOFTWARE.
\end{lstlisting}

\subsection*{Grenzfälle}\label{subsec:normalmodus-grenzfaelle}

\paragraph*{T2.16} Zu kurze Eingabe.
\begin{lstlisting}[escapechar=\%]
Bitte geben Sie ein kodiertes Wort ein:
%\textcolor{fhfarbe}{1}%
Das Wort sollte mindestens einen Buchstaben enthalten!
Bitte geben Sie ein kodiertes Wort ein:
\end{lstlisting}

\paragraph*{T2.17} Eingabe ist zu lang.
\begin{lstlisting}[escapechar=\%]
Bitte geben Sie ein kodiertes Wort ein:
%\textcolor{fhfarbe}{763892732323332240}%
Wort zu lang! Es sind hoechstens 16 Ziffern erlaubt (15 Buchstaben und 1 Wort-Ende)!
Es muss genau ein Leerzeichen (1) oder Punkt (0) am Wortende stehen!
Bitte geben Sie ein kodiertes Wort ein:
\end{lstlisting}


\section{Explizitmodus-Eingabe-Tests}\label{sec:explizitmodus-tests}

In den folgenden Testfällen wird geprüft, ob das Programm bei Verletzung der in Kapitel~\ref{subsec:syntaktische-fehler} syntaktischen Regeln für den Explizitmodus, passend reagiert.
In allen Fällen sollte ein Fehler gemeldet und eine neue Eingabe gefordert werden.

\subsection*{Fehlerfälle}\label{subsec:explizit-fehlerfaelle}

\paragraph*{T3.1} Unerlaubte Zeichen.
\begin{lstlisting}[escapechar=\%]
Bitte Wort im Explizit-Modus eingeben:
%\textcolor{fhfarbe}{3343g7423420}%
Die Eingabe darf nur Ziffern von 0-9 enthalten!
Bitte Wort im Explizit-Modus eingeben:
\end{lstlisting}

\paragraph*{T3.2} Leerzeichen im Wort.
\begin{lstlisting}[escapechar=\%]
Bitte Wort im Explizit-Modus eingeben:
%\textcolor{fhfarbe}{3343 7423420}%
Leerzeichen innerhalb des Wortes sind nicht erlaubt!
Bitte Wort im Explizit-Modus eingeben:
\end{lstlisting}

\paragraph*{T3.3} Gerade Anzahl von Ziffern.
\begin{lstlisting}[escapechar=\%]
Bitte Wort im Explizit-Modus eingeben:
%\textcolor{fhfarbe}{334327423420}%
Es wurde eine gerade Zahl von Ziffern eingeben. Im Explizitmodus werden buchstaben als Paare kodiert, das Abschliessende Symbol (Punkt (0) oder Leerzeichen (1)) wird aber nur einmal eingegeben!
Bitte Wort im Explizit-Modus eingeben:
\end{lstlisting}

\paragraph*{T3.4} Ungültiges Wortende.
\begin{lstlisting}[escapechar=\%]
Bitte Wort im Explizit-Modus eingeben:
%\textcolor{fhfarbe}{33437423423}%
Das Wort wurde mit keinem Gueltigen Zeichen (0 oder 1) abgeschlossen!
Bitte Wort im Explizit-Modus eingeben:
\end{lstlisting}

\paragraph*{T3.5} Unerlaubte Zeichen und Leerzeichen im Wort.
\begin{lstlisting}[escapechar=\%]
Bitte Wort im Explizit-Modus eingeben:
%\textcolor{fhfarbe}{3343g74 23420}%
Die Eingabe darf nur Ziffern von 0-9 enthalten!
Leerzeichen innerhalb des Wortes sind nicht erlaubt!
Bitte Wort im Explizit-Modus eingeben:
\end{lstlisting}

\paragraph*{T3.6} Unerlaubte Zeichen und gerade Anzahl von Ziffern.
\begin{lstlisting}[escapechar=\%]
Bitte Wort im Explizit-Modus eingeben:
%\textcolor{fhfarbe}{3343g743420}%
Die Eingabe darf nur Ziffern von 0-9 enthalten!
Es wurde eine gerade Zahl von Ziffern eingeben. Im Explizitmodus werden buchstaben als Paare kodiert, das Abschliessende Symbol (Punkt (0) oder Leerzeichen (1)) wird aber nur einmal eingegeben!
Bitte Wort im Explizit-Modus eingeben:
\end{lstlisting}

\paragraph*{T3.7} Unerlaubte Zeichen und ungueltiges Wortende.
\begin{lstlisting}[escapechar=\%]
Bitte Wort im Explizit-Modus eingeben:
%\textcolor{fhfarbe}{3343g7423424}%
Die Eingabe darf nur Ziffern von 0-9 enthalten!
Das Wort wurde mit keinem Gueltigen Zeichen (0 oder 1) abgeschlossen!
Bitte Wort im Explizit-Modus eingeben:
\end{lstlisting}

\paragraph*{T3.8} Leerzeichen im Wort und gerade Anzahl von Ziffern.
\begin{lstlisting}[escapechar=\%]
Bitte Wort im Explizit-Modus eingeben:
%\textcolor{fhfarbe}{33432 7423420}%
Leerzeichen innerhalb des Wortes sind nicht erlaubt!
Es wurde eine gerade Zahl von Ziffern eingeben. Im Explizitmodus werden buchstaben als Paare kodiert, das Abschliessende Symbol (Punkt (0) oder Leerzeichen (1)) wird aber nur einmal eingegeben!
Bitte Wort im Explizit-Modus eingeben:
\end{lstlisting}

\paragraph*{T3.9} Leerzeichen im Wort und ungueltiges Wortende.
\begin{lstlisting}[escapechar=\%]
Bitte Wort im Explizit-Modus eingeben:
%\textcolor{fhfarbe}{334374 23423}%
Leerzeichen innerhalb des Wortes sind nicht erlaubt!
Das Wort wurde mit keinem Gueltigen Zeichen (0 oder 1) abgeschlossen!
Bitte Wort im Explizit-Modus eingeben:
\end{lstlisting}

\paragraph*{T3.10} Gerade Anzahl von Ziffern und ungültiges Wortende.
\begin{lstlisting}[escapechar=\%]
Bitte Wort im Explizit-Modus eingeben:
%\textcolor{fhfarbe}{3343742342}%
Es wurde eine gerade Zahl von Ziffern eingeben. Im Explizitmodus werden buchstaben als Paare kodiert, das Abschliessende Symbol (Punkt (0) oder Leerzeichen (1)) wird aber nur einmal eingegeben!
Das Wort wurde mit keinem Gueltigen Zeichen (0 oder 1) abgeschlossen!
Bitte Wort im Explizit-Modus eingeben:
\end{lstlisting}

\paragraph*{T3.11} Unerlaubte Zeichen und Leerzeichen im Wort und gerade Anzahl von Ziffern.
\begin{lstlisting}[escapechar=\%]
Bitte Wort im Explizit-Modus eingeben:
%\textcolor{fhfarbe}{3343g74 234230}%
Die Eingabe darf nur Ziffern von 0-9 enthalten!
Leerzeichen innerhalb des Wortes sind nicht erlaubt!
Es wurde eine gerade Zahl von Ziffern eingeben. Im Explizitmodus werden buchstaben als Paare kodiert, das Abschliessende Symbol (Punkt (0) oder Leerzeichen (1)) wird aber nur einmal eingegeben!
Bitte Wort im Explizit-Modus eingeben:
\end{lstlisting}

\paragraph*{T3.12} Unerlaubte Zeichen und Leerzeichen im Wort und ungültiges Wortende.
\begin{lstlisting}[escapechar=\%]
Bitte Wort im Explizit-Modus eingeben:
%\textcolor{fhfarbe}{3343g74 23423}%
Die Eingabe darf nur Ziffern von 0-9 enthalten!
Leerzeichen innerhalb des Wortes sind nicht erlaubt!
Das Wort wurde mit keinem Gueltigen Zeichen (0 oder 1) abgeschlossen!
Bitte Wort im Explizit-Modus eingeben:
\end{lstlisting}

\paragraph*{T3.13} Gerade Anzahl von Ziffern und Leerzeichen im Wort und ungültiges Wortende.
\begin{lstlisting}[escapechar=\%]
Bitte Wort im Explizit-Modus eingeben:
%\textcolor{fhfarbe}{334374 2342}%
Leerzeichen innerhalb des Wortes sind nicht erlaubt!
Es wurde eine gerade Zahl von Ziffern eingeben. Im Explizitmodus werden buchstaben als Paare kodiert, das Abschliessende Symbol (Punkt (0) oder Leerzeichen (1)) wird aber nur einmal eingegeben!
Das Wort wurde mit keinem Gueltigen Zeichen (0 oder 1) abgeschlossen!
Bitte Wort im Explizit-Modus eingeben:
\end{lstlisting}

\paragraph*{T3.14} Gerade Anzahl von Ziffern und Leerzeichen im Wort und ungültiges Wortende und ungültige Zeichen.
\begin{lstlisting}[escapechar=\%]
Bitte Wort im Explizit-Modus eingeben:
%\textcolor{fhfarbe}{3343z74 2342}%
Die Eingabe darf nur Ziffern von 0-9 enthalten!
Leerzeichen innerhalb des Wortes sind nicht erlaubt!
Es wurde eine gerade Zahl von Ziffern eingeben. Im Explizitmodus werden buchstaben als Paare kodiert, das Abschliessende Symbol (Punkt (0) oder Leerzeichen (1)) wird aber nur einmal eingegeben!
Das Wort wurde mit keinem Gueltigen Zeichen (0 oder 1) abgeschlossen!
Bitte Wort im Explizit-Modus eingeben:
\end{lstlisting}

\paragraph*{T3.15} Alle oben stehenden Fehler hintereinander.
\begin{lstlisting}[escapechar=\%]
Bitte Wort im Explizit-Modus eingeben:
%\textcolor{fhfarbe}{3343g7423420}%
Die Eingabe darf nur Ziffern von 0-9 enthalten!
Bitte Wort im Explizit-Modus eingeben:
%\textcolor{fhfarbe}{3343 7423420}%
Leerzeichen innerhalb des Wortes sind nicht erlaubt!
Bitte Wort im Explizit-Modus eingeben:
%\textcolor{fhfarbe}{334327423420}%
Es wurde eine gerade Zahl von Ziffern eingeben. Im Explizitmodus werden buchstaben als Paare kodiert, das Abschliessende Symbol (Punkt (0) oder Leerzeichen (1)) wird aber nur einmal eingegeben!
Bitte Wort im Explizit-Modus eingeben:
%\textcolor{fhfarbe}{33437423423}%
Das Wort wurde mit keinem Gueltigen Zeichen (0 oder 1) abgeschlossen!
Bitte Wort im Explizit-Modus eingeben:
%\textcolor{fhfarbe}{3343g74 23420}%
Die Eingabe darf nur Ziffern von 0-9 enthalten!
Leerzeichen innerhalb des Wortes sind nicht erlaubt!
Bitte Wort im Explizit-Modus eingeben:
%\textcolor{fhfarbe}{3343g743420}%
Die Eingabe darf nur Ziffern von 0-9 enthalten!
Es wurde eine gerade Zahl von Ziffern eingeben. Im Explizitmodus werden buchstaben als Paare kodiert, das Abschliessende Symbol (Punkt (0) oder Leerzeichen (1)) wird aber nur einmal eingegeben!
Bitte Wort im Explizit-Modus eingeben:
%\textcolor{fhfarbe}{3343g7423424}%
Die Eingabe darf nur Ziffern von 0-9 enthalten!
Das Wort wurde mit keinem Gueltigen Zeichen (0 oder 1) abgeschlossen!
Bitte Wort im Explizit-Modus eingeben:
%\textcolor{fhfarbe}{33432 7423420}%
Leerzeichen innerhalb des Wortes sind nicht erlaubt!
Es wurde eine gerade Zahl von Ziffern eingeben. Im Explizitmodus werden buchstaben als Paare kodiert, das Abschliessende Symbol (Punkt (0) oder Leerzeichen (1)) wird aber nur einmal eingegeben!
Bitte Wort im Explizit-Modus eingeben:
%\textcolor{fhfarbe}{334374 23423}%
Leerzeichen innerhalb des Wortes sind nicht erlaubt!
Das Wort wurde mit keinem Gueltigen Zeichen (0 oder 1) abgeschlossen!
Bitte Wort im Explizit-Modus eingeben:
%\textcolor{fhfarbe}{3343742342}%
Es wurde eine gerade Zahl von Ziffern eingeben. Im Explizitmodus werden buchstaben als Paare kodiert, das Abschliessende Symbol (Punkt (0) oder Leerzeichen (1)) wird aber nur einmal eingegeben!
Das Wort wurde mit keinem Gueltigen Zeichen (0 oder 1) abgeschlossen!
Bitte Wort im Explizit-Modus eingeben:
%\textcolor{fhfarbe}{3343g74 234230}%
Die Eingabe darf nur Ziffern von 0-9 enthalten!
Leerzeichen innerhalb des Wortes sind nicht erlaubt!
Es wurde eine gerade Zahl von Ziffern eingeben. Im Explizitmodus werden buchstaben als Paare kodiert, das Abschliessende Symbol (Punkt (0) oder Leerzeichen (1)) wird aber nur einmal eingegeben!
Bitte Wort im Explizit-Modus eingeben:
%\textcolor{fhfarbe}{3343g74 23423}%
Die Eingabe darf nur Ziffern von 0-9 enthalten!
Leerzeichen innerhalb des Wortes sind nicht erlaubt!
Das Wort wurde mit keinem Gueltigen Zeichen (0 oder 1) abgeschlossen!
Bitte Wort im Explizit-Modus eingeben:
%\textcolor{fhfarbe}{334374 2342}%
Leerzeichen innerhalb des Wortes sind nicht erlaubt!
Es wurde eine gerade Zahl von Ziffern eingeben. Im Explizitmodus werden buchstaben als Paare kodiert, das Abschliessende Symbol (Punkt (0) oder Leerzeichen (1)) wird aber nur einmal eingegeben!
Das Wort wurde mit keinem Gueltigen Zeichen (0 oder 1) abgeschlossen!
Bitte Wort im Explizit-Modus eingeben:
%\textcolor{fhfarbe}{3343z74 2342}%
Die Eingabe darf nur Ziffern von 0-9 enthalten!
Leerzeichen innerhalb des Wortes sind nicht erlaubt!
Es wurde eine gerade Zahl von Ziffern eingeben. Im Explizitmodus werden buchstaben als Paare kodiert, das Abschliessende Symbol (Punkt (0) oder Leerzeichen (1)) wird aber nur einmal eingegeben!
Das Wort wurde mit keinem Gueltigen Zeichen (0 oder 1) abgeschlossen!
Bitte Wort im Explizit-Modus eingeben:
%\textcolor{fhfarbe}{33437423420}%
FISCH
Wort in Ordnung? ja (*) / nein (#)
%\textcolor{fhfarbe}{*}%
Das Wort FISCH wurde mit Code 34724 im Woerterbuch abgespeichert!
Eingegebener Satz:
FISCH.
\end{lstlisting}

\subsection*{Grenzfälle}\label{subsec:explizit-grenzfaelle}

\paragraph*{T3.16} Grenzfall: Zu kurze Eingabe.
\begin{lstlisting}[escapechar=\%]
Bitte Wort im Explizit-Modus eingeben:
%\textcolor{fhfarbe}{1}%
Das Wort sollte mindestens einen Buchstaben enthalten!
Bitte Wort im Explizit-Modus eingeben:
\end{lstlisting}

\paragraph*{T3.17} Grenzfall: Eingabe ist zu lang.
\begin{lstlisting}[escapechar=\%]
Bitte Wort im Explizit-Modus eingeben:
%\textcolor{fhfarbe}{334374234222222222222222222233333333330}%
Wort zu lang! Es sind hoechstens 31 Ziffern erlaubt (15 Buchstabenpaare und 1 Wort-Ende)!
Es wurde eine gerade Zahl von Ziffern eingeben. Im Explizitmodus werden buchstaben als Paare kodiert, das Abschliessende Symbol (Punkt (0) oder Leerzeichen (1)) wird aber nur einmal eingegeben!
Das Wort wurde mit keinem Gueltigen Zeichen (0 oder 1) abgeschlossen!
Bitte Wort im Explizit-Modus eingeben:
\end{lstlisting}


\section{Semantik Explizitmodus-Tests}\label{sec:semantik-explizitmodus-tests}
In den folgenden Tests werden Explizit-Eingaben geprüft, die den syntaktischen Vorgaben genügen, aber nicht dem strukturellen Aufbau der Handytastatur entsprechen.

\subsection*{Fehlerfälle}\label{subsec:sematik-fehler}

\paragraph*{T4.1} Zugriff auf nicht existierenden Buchstaben 64.
\begin{lstlisting}[escapechar=\%]
Bitte Wort im Explizit-Modus eingeben:
%\textcolor{fhfarbe}{64218142320}%
64 ist kein gueltiger Buchstabe!
Bitte Wort im Explizit-Modus eingeben:
\end{lstlisting}

\paragraph*{T4.2} Index 65 außerhalb der maximalen Tastenlänge.
\begin{lstlisting}[escapechar=\%]
Bitte Wort im Explizit-Modus eingeben:
%\textcolor{fhfarbe}{65218142320}%
Buchstabenindex 5 des 01-ten Buchstaben zu hoch! Tasten haben maximal 4 Eintraege!
Bitte Wort im Explizit-Modus eingeben:
\end{lstlisting}

\paragraph*{T4.3} Mehrere ungültige Buchstaben.
\begin{lstlisting}[escapechar=\%]
Bitte Wort im Explizit-Modus eingeben:
%\textcolor{fhfarbe}{64248142320}%
64 ist kein gueltiger Buchstabe!
24 ist kein gueltiger Buchstabe!
Bitte Wort im Explizit-Modus eingeben:
\end{lstlisting}

\paragraph*{T4.4} Ungültiger Buchstabe und Index größer 4.
\begin{lstlisting}[escapechar=\%]
Bitte Wort im Explizit-Modus eingeben:
%\textcolor{fhfarbe}{65118142320}%
Buchstabenindex 5 des 01-ten Buchstaben zu hoch! Tasten haben maximal 4 Eintraege!
11 ist kein gueltiger Buchstabe!
Bitte Wort im Explizit-Modus eingeben:
\end{lstlisting}

\paragraph*{T4.5} Mehrere Indizes größer als 4.
\begin{lstlisting}[escapechar=\%]
Bitte Wort im Explizit-Modus eingeben:
%\textcolor{fhfarbe}{65218142390}%
Buchstabenindex 5 des 01-ten Buchstaben zu hoch! Tasten haben maximal 4 Eintraege!
Buchstabenindex 9 des 05-ten Buchstaben zu hoch! Tasten haben maximal 4 Eintraege!
Bitte Wort im Explizit-Modus eingeben:
\end{lstlisting}

\subsection*{Grenzfälle}\label{subsec:semantik-grenz}
\paragraph*{T4.6} Wort wird nicht dem Wörterbuch hinzugefügt, da es bereits voll ist. Es wird trotzdem dem Satz angehängt.
\begin{lstlisting}[escapechar=\%]
Bitte Wort im Explizit-Modus eingeben:
%\textcolor{fhfarbe}{61218142320}%
MATHE
Wort in Ordnung? ja (*) / nein (#)
%\textcolor{fhfarbe}{*}%
Das Wort konnte nicht abgespeichert werden, weil das Woerterbuch voll ist!
Eingegebener Satz:
MATHE.
\end{lstlisting}


\section{Wortvorschläge-Tests}\label{sec:wortvorschläge-tests}
\subsection*{Normalfälle}\label{subsec:vorschlag-normalfaelle}
Wörter, die einem eingegebenen T9-Code entsprechen, sollen in der Reihenfolge ihrer Häufigkeit ausgegeben werden.
\paragraph*{T5.1} Ausgabe der Wörter in richtiger Reihenfolge.
Das Wörterbuch enthält mehrere Wörter zu einem T9-Code (Zur Übersicht wurden Wörter gewählt, die nicht zum T9-Code passen):
\begin{center}
    \fbox{\begin{minipage}{15em}
              2345 DREI 3 \\
              2345 ZWEI 9 \\
              2345 VIER 2 \\
              2345 FUENF 1 \\
              2345 EINS 13
    \end{minipage}}
\end{center}
\begin{lstlisting}[escapechar=\%]
Moechten Sie ein externes Woerterbuch einlesen? ja (*) / nein (#)
%\textcolor{fhfarbe}{*}%
Bitte geben sie den Dateinamen des Woerterbuchs ein.
%\textcolor{fhfarbe}{TestWBs/WBuch-Reihenfolge.txt}%
Woerterbuch erfolgreich eingelesen.
Beginnen Sie einen neuen Satz oder beenden Sie das Programm mit (0).
Bitte geben Sie ein kodiertes Wort ein:
%\textcolor{fhfarbe}{23450}%
EINS
Wort in Ordnung? ja (*) / nein (#)
%\textcolor{fhfarbe}{\#}%
ZWEI
Wort in Ordnung? ja (*) / nein (#)
%\textcolor{fhfarbe}{\#}%
DREI
Wort in Ordnung? ja (*) / nein (#)
%\textcolor{fhfarbe}{\#}%
VIER
Wort in Ordnung? ja (*) / nein (#)
%\textcolor{fhfarbe}{\#}%
FUENF
Wort in Ordnung? ja (*) / nein (#)
%\textcolor{fhfarbe}{*}%
Eingegebener Satz:
FUENF.
\end{lstlisting}


\section{Wörterbücher-Tests}\label{sec:wörterbücher-tests}

\subsection*{Normalfälle}\label{subsec:wbuch-normalfaelle}
\paragraph*{T6.1} 6.1: Einlesen eines existierenden Wörterbuchs funktioniert fehlerfrei
\begin{lstlisting}[escapechar=\%]
Moechten Sie ein externes Woerterbuch einlesen? ja (*) / nein (#)
%\textcolor{fhfarbe}{*}%
Bitte geben Sie den Dateinamen des Woerterbuchs ein.
%\textcolor{fhfarbe}{TestWBs/WBuch-correct.txt}%
Woerterbuch erfolgreich eingelesen.
\end{lstlisting}

\paragraph*{T6.1} 6.1.1: Alle Einträge des Wörterbuchs sind verfügbar:
\begin{lstlisting}[escapechar=\%]
Moechten Sie ein externes Woerterbuch einlesen? ja (*) / nein (#)
%\textcolor{fhfarbe}{*}%
Bitte geben Sie den Dateinamen des Woerterbuchs ein.
%\textcolor{fhfarbe}{TestWBs/WBuch-correct.txt}%
Woerterbuch erfolgreich eingelesen.
Beginnen Sie einen neuen Satz oder beenden Sie das Programm mit (0).
Bitte geben Sie ein kodiertes Wort ein:
%\textcolor{fhfarbe}{763892731}%
SOFTWARE
Wort in Ordnung? ja (*) / nein (#)
%\textcolor{fhfarbe}{*}%
Bitte geben Sie ein kodiertes Wort ein:
%\textcolor{fhfarbe}{96781}%
WORT
Wort in Ordnung? ja (*) / nein (#)
%\textcolor{fhfarbe}{*}%
Bitte geben Sie ein kodiertes Wort ein:
%\textcolor{fhfarbe}{3371}%
DER
Wort in Ordnung? ja (*) / nein (#)
%\textcolor{fhfarbe}{*}%
Bitte geben Sie ein kodiertes Wort ein:
%\textcolor{fhfarbe}{3370}%
DER
Wort in Ordnung? ja (*) / nein (#)
%\textcolor{fhfarbe}{\#}%
FES
Wort in Ordnung? ja (*) / nein (#)
%\textcolor{fhfarbe}{*}%
Eingegebener Satz:
SOFTWARE WORT DER FES.
\end{lstlisting}

\subsection*{Fehlerfälle}\label{subsec:wbuch-fehlerfaelle}
\paragraph*{T6.2} Einlesen eines korrupten Wörterbuchs erzeugt Fehlermeldung.
\begin{lstlisting}[escapechar=\%]
Moechten Sie ein externes Woerterbuch einlesen? ja (*) / nein (#)
%\textcolor{fhfarbe}{*}%
Bitte geben Sie den Dateinamen des Woerterbuchs ein.
%\textcolor{fhfarbe}{TestWBs/WBuch-incorrect.txt}%
Der 1-te Satz der Datei liegt nicht im richtigen Format vor!
Wenn mit einem beschaedigten Woerterbuch gearbeitet wird, kann es zur Laufzeit zu Fehlern kommen! Beheben Sie die Syntaxfehler, lesen Sie ein anderes Woerterbuch ein oder fahren Sie fort ohne Woerterbuch.
\end{lstlisting}

\paragraph*{T6.3} Einlesen eines nicht existierenden Wörterbuchs erzeugt Fehlermeldung.
\begin{lstlisting}[escapechar=\%]
Moechten Sie ein externes Woerterbuch einlesen? ja (*) / nein (#)
%\textcolor{fhfarbe}{*}%
Bitte geben Sie den Dateinamen des Woerterbuchs ein.
%\textcolor{fhfarbe}{IDoNotExist.txt}%
Kein Woerterbuch geladen.
\end{lstlisting}

\paragraph*{T6.4} Einlesen eines leeren Wörterbuchs erzeugt Fehlermeldung.
\begin{lstlisting}[escapechar=\%]
Moechten Sie ein externes Woerterbuch einlesen? ja (*) / nein (#)
%\textcolor{fhfarbe}{*}%
Bitte geben Sie den Dateinamen des Woerterbuchs ein.
%\textcolor{fhfarbe}{TestWBs/WBuch-empty.txt}%
Woerterbuch ist leer.
Woerterbuch erfolgreich eingelesen.
\end{lstlisting}


\subsection*{Sonderfälle}\label{subsec:wbuch-sonderfaelle}
\paragraph*{T6.5} Nach Programmende enthält Wörterbuch sowohl eingelesene als auch neue Einträge.
\begin{lstlisting}[escapechar=\%]
Moechten Sie ein externes Woerterbuch einlesen? ja (*) / nein (#)
%\textcolor{fhfarbe}{*}%
Bitte geben Sie den Dateinamen des Woerterbuchs ein.
%\textcolor{fhfarbe}{TestWBs/WBuch-correct.txt}%
Woerterbuch erfolgreich eingelesen.
Beginnen Sie einen neuen Satz oder beenden Sie das Programm mit (0).
Bitte geben Sie ein kodiertes Wort ein:
%\textcolor{fhfarbe}{2581}%
ALT
Wort in Ordnung? ja (*) / nein (#)
%\textcolor{fhfarbe}{*}%
Bitte geben Sie ein kodiertes Wort ein:
%\textcolor{fhfarbe}{4241}%
Im Woerterbuch wurde kein passender Eintrag gefunden!
Bitte Wort im Explizit-Modus eingeben:
%\textcolor{fhfarbe}{4323421}%
ICH
Wort in Ordnung? ja (*) / nein (#)
%\textcolor{fhfarbe}{*}%
Das Wort ICH wurde mit Code 424 im Woerterbuch abgespeichert!
Bitte geben Sie ein kodiertes Wort ein:
%\textcolor{fhfarbe}{2461}%
Im Woerterbuch wurde kein passender Eintrag gefunden!
Bitte Wort im Explizit-Modus eingeben:
%\textcolor{fhfarbe}{2243621}%
BIN
Wort in Ordnung? ja (*) / nein (#)
%\textcolor{fhfarbe}{*}%
Das Wort BIN wurde mit Code 246 im Woerterbuch abgespeichert!
Bitte geben Sie ein kodiertes Wort ein:
%\textcolor{fhfarbe}{6380}%
Im Woerterbuch wurde kein passender Eintrag gefunden!
Bitte Wort im Explizit-Modus eingeben:
%\textcolor{fhfarbe}{6232820}%
NEU
Wort in Ordnung? ja (*) / nein (#)
%\textcolor{fhfarbe}{*}%
Das Wort NEU wurde mit Code 638 im Woerterbuch abgespeichert!
Eingegebener Satz:
ALT ICH BIN NEU.
\end{lstlisting}

\begin{center}
    \fbox{\begin{minipage}{15em}
              258 ALT 1 \\
              246 BIN 1 \\
              424 ICH 1 \\
              638 NEU 1
    \end{minipage}}
    \captionof{figure}{Wörterbuch nach Test 6.5}
    \label{fig:wbuch4}
\end{center}
\subsection*{Grenzfall}\label{subsec:wbuch-grenzfaelle}
\paragraph*{T6.2} Grenzfall: Textdatei hat mehr Einträge als interne Datenstruktur.
Erwartet wird eine Meldung, es wird aber keine erneute Eingabe gefordert.
\begin{lstlisting}[escapechar=\%]
Moechten Sie ein externes Woerterbuch einlesen? ja (*) / nein (#)
%\textcolor{fhfarbe}{*}%
Bitte geben Sie den Dateinamen des Woerterbuchs ein.
%\textcolor{fhfarbe}{TestWBs/WBuch-401-Eintraege.txt}%
Alle Eintraege ab dem  401-ten Satz der Datei wurden nicht eingelesen, da das Woerterbuch voll ist.
Woerterbuch erfolgreich eingelesen.
\end{lstlisting}

\section{Sonderfälle}\label{sec:sonderfall-tests}
Nun werden die bereits beschriebenen Sonderfälle (Kapitel~\ref{subsec:sonderfaelle}) getestet.

\subsection*{Normalfälle}\label{subsec:sonder-normalfaelle}
\paragraph*{T7.1} Der Anwender gibt im Explizitmodus etwas ein, was nicht der initialen Eingabe im normalen Modus entspricht.
Erwartet wird eine Meldung, die darauf hinweist und das Speichern des Wortes aus der Explizit-Eingabe.
\begin{lstlisting}[escapechar=\%]
Bitte geben Sie ein kodiertes Wort ein:
%\textcolor{fhfarbe}{3370}%
Im Woerterbuch wurde kein passender Eintrag gefunden!
Bitte Wort im Explizit-Modus eingeben:
%\textcolor{fhfarbe}{325332332162810}%
ELEFANT
Wort in Ordnung? ja (*) / nein (#)
%\textcolor{fhfarbe}{*}%
Das explizit eingegebene Wort entspricht nicht der originalen Eingabe.
Das Wort ELEFANT wurde mit Code 3533268 im Woerterbuch abgespeichert!
Eingegebener Satz:
ELEFANT.
\end{lstlisting}

\paragraph*{T7.2} Der Anwender gibt im Explizitmodus etwas ein, was nicht der initialen Eingabe im normalen Modus entspricht.
Erwartet wird eine Meldung, die darauf hinweist.
Das Explizit-Wort ist bereits im Wörterbuch vorhanden.
Deshalb wird nur die Häufigkeit erhöht.
\begin{center}
    \fbox{\begin{minipage}{15em}
              3533268 ELEFANT 1
    \end{minipage}}
    \captionof{figure}{Wörterbuch vor Test 7.2.}
    \label{fig:wbuch7}
\end{center}

\begin{lstlisting}[escapechar=\%]
Moechten Sie ein externes Woerterbuch einlesen? ja (*) / nein (#)
%\textcolor{fhfarbe}{*}%
Bitte geben Sie den Dateinamen des Woerterbuchs ein.
%\textcolor{fhfarbe}{TestWBs/WBuch-Sonderfall.txt}%
Woerterbuch erfolgreich eingelesen.
Beginnen Sie einen neuen Satz oder beenden Sie das Programm mit (0).
Bitte geben Sie ein kodiertes Wort ein:
%\textcolor{fhfarbe}{3370}%
Im Woerterbuch wurde kein passender Eintrag gefunden!
Bitte Wort im Explizit-Modus eingeben:
%\textcolor{fhfarbe}{325332332162810}%
ELEFANT
Wort in Ordnung? ja (*) / nein (#)
%\textcolor{fhfarbe}{*}%
Das explizit eingegebene Wort entspricht nicht der originalen Eingabe.
Wort war bereits im Woerterbuch vorhanden.
Eingegebener Satz:
ELEFANT.
Beginnen Sie einen neuen Satz oder beenden Sie das Programm mit (0).
Bitte geben Sie ein kodiertes Wort ein:
%\textcolor{fhfarbe}{0}%
Programm Ende.
\end{lstlisting}

\begin{center}
    \fbox{\begin{minipage}{15em}
              3533268 ELEFANT 2
    \end{minipage}}
    \captionof{figure}{Wörterbuch nach Test 7.2.}
    \label{fig:wbuch8}
\end{center}