\chapter{Entwicklungsumgebung}\label{ch:entwicklungsumgebung}

Das Programm wurde mithilfe der OpenCobolIDE (\url{https://launchpad.net/cobcide/+download}) in der Version 4.7.6 geschrieben.
Dabei handelt es sich um eine leichtgewichtige COBOL Entwicklungsumgebung, die als Compiler GnuCOBOL 2.0.0 (\url{https://sourceforge.net/projects/gnucobol/}) verwendet.
Der GnuCOBOL-Compiler übersetzt den COBOL-Source-Code in ein C-Programm und erzeugt daraus, mit einem nativen C-Compiler, eine ausführbare Datei.

Im Entwicklungsprozess wurde zur Versionsverwaltung GitHub (\url{https://github.com/}) verwendet.
Die dort angebotenen Remote-Repositories ermöglichen eine effiziente Zusammenarbeit im Team.

Alle Entwicklungsschritte wurden auf Systemen mit Windows 10 Betriebssystem (\url{https://www.microsoft.com/de-de/software-download/windows10}) durchgeführt.

\begin{figure}[htb]
    \centering
    \begin{minipage}{.5\textwidth}
        \centering
        \includegraphics[width=.4\linewidth]{images/opencobol-logo}
    \end{minipage}%
    \begin{minipage}{.5\textwidth}
        \centering
        \includegraphics[width=.4\linewidth]{images/Gnu-COBOL}
    \end{minipage}
    \caption{Logos von OpenCobolIDE und GnuCOBOL.}
    \label{fig:cobol-logos}
\end{figure}

\begin{figure}[htb]
    \centering
    \includegraphics[width=.15\linewidth]{images/GitHub-Logo}
    \caption{
        GitHub-Logo.
    }
    \label{fig:github}
\end{figure}
