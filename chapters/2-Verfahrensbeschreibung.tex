\chapter{Verfahrensbeschreibung}\label{ch:verfahrensbeschreibung}


\section{Gesamtsystem}\label{sec:gesamtsystem}
Das System arbeitet nach dem \textbf{E}ingabe, \textbf{V}erarbeitung, \textbf{A}usgabe-Prinzip, kurz EVA.
EVA ist ein Grundprinzip der Datenverarbeitung, bei welchem die drei Schritte sequenziell durchlaufen werden.
In diesem Fall lassen sich die drei Fälle nur von außen voneinander Trennen. So wird die Eingabe Zeilenweise vollzogen. Jede eingelesene Zeile wird sofort verarbeitet und sodass eine Ausgabezeile erstellt und ausgegeben wird. Jeder Schritt besteht also aus allen drei Phasen.\\

\subsection{Eingabe}\label{subsec:eingabe}
Es gibt keine direkte Eingabe, da das Programm durch das einlesen einer Datei gesteuert wird.

\subsection{Verarbeitung}\label{subsec:verarbeitung}
Die Verarbeitung nutzt das Prinzip des Gruppenwechsels. Dazu wird zunächst die KundenID eingelesen und gespeichert. Selbiges geschiet mit der LeistungsID. Solange die neu eingelesen LeistungsID gleich der vorherigen ist, wird die Anzahl der Leistungen erhöht. Sobald die LeistungsID wechselt, wird die vorherige Leistung in die Rechnung geschrieben und die Anzahl wieder auf 1 gesetzt.\\


Erfolgte die Eingabe über den Explizitmodus wird zuerst überprüft, ob sie legal ist.
~\ZB~dürfen keine Symbole, sondern nur Ziffern, eingegeben werden und die Eingabe muss mit einem ␣ (Taste 1) oder . (Taste 0) enden.
Siehe hierzu~\nameref{subsec:syntaktische-fehler}.\\
Außerdem erkennt das Programm Unregelmäßigkeiten und informiert den Nutzer~\zB~darüber, dass die Eingabe, die er zuvor im Normalmodus getätigt hat, nicht mit dem eingegebenen Zahlencode im Explizitmodus übereinstimmt.
In diesem speziellen Fall wird das in expliziter Form eingegebene Wort trotzdem gespeichert.\\
Ist die Eingabe korrekt und legal, wird noch überprüft, ob das eingegebene Wort schon im Wörterbuch existiert und falls dies der Fall ist, lediglich die Häufigkeit um eins erhöht.
Sollten die Überprüfungen alle zusagen, wird das Wort im Wörterbuch gespeichert und der Nutzer gelangt wieder in den Normalmodus.\\
Das Abspeichern im Wörterbuch folgt einem Schema, wobei jedes Wort in einer~.txt-Datei eine Zeile erhält.
Jede Zeile besteht aus drei Spalten, getrennt mit einem Leerzeichen.
Die erste Spalte enthält den Zifferncode für das Wort, die zweite das Wort selbst und die dritte Spalte die Häufigkeit des Worts.\\
Der Eintrag HUT sieht in der~.txt-Datei \zB~wie folgt aus:\\

\begin{centering}
    488 HUT 1\\
\end{centering}

\subsection{Ausgabe}\label{subsec:ausgabe}
Ausgaben erscheinen ebenfalls in dem Fenster der Kommandozeile und informieren den Nutzer über die Verarbeitung seiner Eingaben.
Sollte er beispielsweise einen Satz beendet haben, wird der ganze Satz angezeigt.
Auch Fehlermeldungen werden in der Konsole angezeigt, um den Nutzer auf falsche Eingaben oder Verarbeitung hinzuweisen.
Das angelegte Wörterbuch ist ebenfalls eine Form der Ausgabe und kann im bin/ Verzeichnis eingesehen werden.

\subsection{Datenstrukturen}\label{subsec:datenstrukt}
Zum effizienten Arbeiten mit dem Wörterbuch wird die Eingabedatei in eine COBOL-Tabelle geschrieben.
Diese wird mit einem Spezialindex definiert, um das Verwenden der \glqq SEARCH\grqq{}-Funktion zu ermöglichen.
Am Programmende wird die Tabelle dann wieder in eine Datei geschrieben.

Auch das Tastenfeld zur Expliziteingabe wird als Tabelle realisiert.
Jeder Tabelleneintrag repräsentiert dabei eine Taste.
So kann mit zwei Indizes elegant auf einzelne Buchstaben zugegriffen werden.

Im gesamten Programm wird hauptsächlich mit alphanumerischen Feldern gearbeitet.
Dadurch können dem Anwender sehr ausführliche Fehlermeldungen angezeigt werden.

