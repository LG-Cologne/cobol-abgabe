\chapter{Verfahrensbeschreibung}\label{ch:verfahrensbeschreibung}


\section{Gesamtsystem}\label{sec:gesamtsystem}
Das System arbeitet nach dem \textbf{E}ingabe, \textbf{V}erarbeitung, \textbf{A}usgabe-Prinzip, kurz EVA.
EVA ist ein Grundprinzip der Datenverarbeitung, bei welchem die drei Schritte sequenziell durchlaufen werden.
Das Prinzip ist im Normalfall zustandslos, weshalb dieses System nicht dem reinen EVA-Prinzip entspricht.
Es ist mehr eine leichte Abwandlung dessen, da Eingaben, zu unterschiedlichen Zeitpunkten, zu einer anderen Ausgabe führen können.

\subsection{Eingabe}\label{subsec:eingabe}
Es gibt zwei unterschiedliche Eingabemodi.
Sowohl der Explizit- als auch der Normalmodus nehmen über die Kommandozeile Eingaben entgegen.
Der Normalmodus ist für die Eingabe von bereits bekannten Wörtern im Wörterbuch gedacht, wohingegen der Explizitmodus neue, unbekannte Wörter entgegennimmt.
Einen Spezialfall bietet das Einlesen eines bereits vorhandenen Wörterbuchs.
Es lässt sich zu Beginn des Programmstarts ein Wörterbuch einlesen, um das vom Programm genutzte Wörterbuch zu befüllen.
Der Nutzer muss das externe Wörterbuch am besten in den Ordner legen, in welchem sich die~.exe-Datei befindet und beim Start des Programms den Namen der Datei angeben.
Ansonsten kann der Nutzer auch einen relativen bzw. absoluten Pfad angeben.

\subsection{Verarbeitung}\label{subsec:verarbeitung}
Je nach Eingabemodus wird die Nutzereingabe unterschiedlich verarbeitet.
Ist sie über den Normalmodus passiert, wird im Wörterbuch ein passendes Mapping zum eingegebenen Zahlencode gesucht.
Je nach den zuvor definierten~\hyperref[enm:faelle]{Fällen}~wird unterschiedlich vorgegangen.\\
Im~\ref{itm:fall-nicht-existent} wird der Nutzer automatisch in den Explizitmodus geleitet.\\
Im~\ref{itm:fall-eins-existent} wird der Nutzer durch Eingabeaufforderung gefragt, ob das gefundene Wort gewünscht ist.
Falls \textit{Ja} wird das Wort dem aktuellen Satz hinzugefügt, falls \textit{Nein} wird der Nutzer automatisch in den Explizitmodus geleitet.\\
Im letzten~\ref{itm:fall-mehrere-existent} wird für jedes der gefundenen Wörter abgefragt, ob es das gewünschte ist, bis der Nutzer das erste Mal mit einem \textit{Ja} antwortet.
Sollte es keine passenden Wörter mehr geben, da der Nutzer durchgehend mit \textit{Nein} geantwortet hat, wird der Nutzer auch hier in den Explizitmodus geleitet.\\

Erfolgte die Eingabe über den Explizitmodus wird zuerst überprüft, ob sie legal ist.
~\ZB~dürfen keine Symbole, sondern nur Ziffern, eingegeben werden und die Eingabe muss mit einem ␣ (Taste 1) oder . (Taste 0) enden.
Siehe hierzu~\nameref{subsec:syntaktische-fehler}.\\
Außerdem erkennt das Programm Unregelmäßigkeiten und informiert den Nutzer~\zB~darüber, dass die Eingabe, die er zuvor im Normalmodus getätigt hat, nicht mit dem eingegebenen Zahlencode im Explizitmodus übereinstimmt.
In diesem speziellen Fall wird das in expliziter Form eingegebene Wort trotzdem gespeichert.\\
Ist die Eingabe korrekt und legal, wird noch überprüft, ob das eingegebene Wort schon im Wörterbuch existiert und falls dies der Fall ist, lediglich die Häufigkeit um eins erhöht.
Sollten die Überprüfungen alle zusagen, wird das Wort im Wörterbuch gespeichert und der Nutzer gelangt wieder in den Normalmodus.\\
Das Abspeichern im Wörterbuch folgt einem Schema, wobei jedes Wort in einer~.txt-Datei eine Zeile erhält.
Jede Zeile besteht aus drei Spalten, getrennt mit einem Leerzeichen.
Die erste Spalte enthält den Zifferncode für das Wort, die zweite das Wort selbst und die dritte Spalte die Häufigkeit des Worts.\\
Der Eintrag HUT sieht in der~.txt-Datei \zB~wie folgt aus:\\

\begin{centering}
    488 HUT 1\\
\end{centering}

\subsection{Ausgabe}\label{subsec:ausgabe}
Ausgaben erscheinen ebenfalls in dem Fenster der Kommandozeile und informieren den Nutzer über die Verarbeitung seiner Eingaben.
Sollte er beispielsweise einen Satz beendet haben, wird der ganze Satz angezeigt.
Auch Fehlermeldungen werden in der Konsole angezeigt, um den Nutzer auf falsche Eingaben oder Verarbeitung hinzuweisen.
Das angelegte Wörterbuch ist ebenfalls eine Form der Ausgabe und kann im bin/ Verzeichnis eingesehen werden.

\subsection{Datenstrukturen}\label{subsec:datenstrukt}
Zum effizienten Arbeiten mit dem Wörterbuch wird die Eingabedatei in eine COBOL-Tabelle geschrieben.
Diese wird mit einem Spezialindex definiert, um das Verwenden der \glqq SEARCH\grqq{}-Funktion zu ermöglichen.
Am Programmende wird die Tabelle dann wieder in eine Datei geschrieben.

Auch das Tastenfeld zur Expliziteingabe wird als Tabelle realisiert.
Jeder Tabelleneintrag repräsentiert dabei eine Taste.
So kann mit zwei Indizes elegant auf einzelne Buchstaben zugegriffen werden.

Im gesamten Programm wird hauptsächlich mit alphanumerischen Feldern gearbeitet.
Dadurch können dem Anwender sehr ausführliche Fehlermeldungen angezeigt werden.

