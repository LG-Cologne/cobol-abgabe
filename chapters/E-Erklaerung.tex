\chapter{Erklärung}\label{ch:erklaerung}

Hiermit versichere ich, dass ich die vorliegende Arbeit mit dem Thema
\begin{quote}
    \textit{\titleDocument}
\end{quote}
selbstständig verfasst und keine anderen als die angegebenen Quellen und Hilfsmittel benutzt habe.
Alle Ausführungen, die anderen Schriften wörtlich oder sinngemäß entnommen wurden, sind kenntlich gemacht und die Arbeit ist in gleicher oder ähnlicher Fassung noch nicht Bestandteil einer Studien- oder Prüfungsleistung.
\\
\\
Mein Beitrag zur Abgabe:
\begin{itemize}[noitemsep]
    \item Ich habe große Teile des Benutzerdialogs implementiert.\\
    COBOL-Paragrafen: \glqq Benutzer-Dialog\grqq{},  \glqq Ermittle-Wort\grqq{},  \glqq Wort-Auswahl\grqq{},  \glqq Finde-Moeglichkeiten\grqq{}, \glqq Sortiere-Nach-Haeuf\grqq{}, \glqq Explizit-Eingabe\grqq{}, \glqq Suche-Wort-In-WBuch\grqq{} und \glqq Konstruiere-Wort\grqq{}
    \item Kapitel der Dokumentation: ~\ref{sec:fehlerarten}, ~\ref{sec:fehlerbehandlung}, ~\ref{subsec:datenstrukt}, ~\ref{sec:definierte-tests}, ~\ref{sec:explizitmodus-tests}, ~\ref{sec:semantik-explizitmodus-tests},
    ~\ref{sec:wortvorschläge-tests}, ~\ref{sec:sonderfall-tests}, ~\ref{ch:entwicklungsumgebung}, ~\ref{ch:verwendete-hilfsmittel}
    \item automatische Tests: TestExplizitEingabe.cmd, TestFunktionalitaet.cmd, TestReihenfolgeWortvorschlag.cmd, TestSemantikExplizitEingabe.cmd und TestSonderfall.cmd
\end{itemize}
\vspace*{2cm}

\begingroup
\setlength{\parindent}{0pt} % keine Einrückung bei neuen Absätzen in diesem Bereich

\locationDocument, den \dateDocument
\bigskip
\bigskip

% gewünschte Breite der Unterschriftslinie
\newlength{\widthbox}
\settowidth{\widthbox}{\locationDocument, den \dateDocument}

\makebox[\widthbox]{\hrulefill}\\
Ben Pietsch
\endgroup
\newpage

Hiermit versichere ich, dass ich die vorliegende Arbeit mit dem Thema
\begin{quote}
    \textit{\titleDocument}
\end{quote}
selbstständig verfasst und keine anderen als die angegebenen Quellen und Hilfsmittel benutzt habe.
Alle Ausführungen, die anderen Schriften wörtlich oder sinngemäß entnommen wurden, sind kenntlich gemacht und die Arbeit ist in gleicher oder ähnlicher Fassung noch nicht Bestandteil einer Studien- oder Prüfungsleistung.
\\
\\
Mein Beitrag zur Abgabe:
\begin{itemize}[noitemsep]
    \item Validierung der Benutzereingaben:\\
    COBOL-Paragrafen: \glqq Pruefe-Eingabe\grqq{} und \glqq Pruefe-Explizit-Eingabe\grqq{}
    \item Kapitel der Dokumentation: ~\ref{ch:programmbeschreibung}, ~\ref{ch:zusammenfassung-und-ausblick},
    \item automatische Tests: TestT9-Eingabe.cmd
\end{itemize}
\vspace*{2cm}

\begingroup
\setlength{\parindent}{0pt} % keine Einrückung bei neuen Absätzen in diesem Bereich

\locationDocument, den \dateDocument
\bigskip
\bigskip

% gewünschte Breite der Unterschriftslinie
%\newlength{\widthbox}
\settowidth{\widthbox}{\locationDocument, den \dateDocument}

\makebox[\widthbox]{\hrulefill}\\
Natalie Fritzen
\endgroup
\newpage

Hiermit versichere ich, dass ich die vorliegende Arbeit mit dem Thema
\begin{quote}
    \textit{\titleDocument}
\end{quote}
selbstständig verfasst und keine anderen als die angegebenen Quellen und Hilfsmittel benutzt habe.
Alle Ausführungen, die anderen Schriften wörtlich oder sinngemäß entnommen wurden, sind kenntlich gemacht und die Arbeit ist in gleicher oder ähnlicher Fassung noch nicht Bestandteil einer Studien- oder Prüfungsleistung.
\\
\\
Mein Beitrag zur Abgabe:
\begin{itemize}[noitemsep]
    \item Einlesen von externen Wörterbüchern und das sortieren und schreiben des internen Wörterbuchs.\\
    COBOL-Paragrafen: \glqq Auswahl-WBuch\grqq{},  \glqq Einlesen-WBuch.\grqq{},  \glqq Lese-Satz\grqq{} und \glqq Schreibe-WBuch-Sortiert\grqq{}
    \item Kapitel der Dokumentation: ~\ref{sec:interpretation-der-aufgabe}, ~\ref{sec:anforderung-an-das-programm}, ~\ref{subsec:eingabe}, ~\ref{subsec:verarbeitung}, ~\ref{subsec:ausgabe}, ~\ref{sec:wörterbücher-tests}, ~\ref{ch:benutzeranleitung}
    \item automatische Tests: TestAlle.cmd, und TestWBuchEinlesen.cmd
\end{itemize}
\vspace*{2cm}

\begingroup
\setlength{\parindent}{0pt} % keine Einrückung bei neuen Absätzen in diesem Bereich

\locationDocument, den \dateDocument
\bigskip
\bigskip

% gewünschte Breite der Unterschriftslinie
%\newlength{\widthbox}
\settowidth{\widthbox}{\locationDocument, den \dateDocument}

\makebox[\widthbox]{\hrulefill}\\
Leonhard Keßler
\endgroup