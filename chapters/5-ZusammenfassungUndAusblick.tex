\chapter{Zusammenfassung und Ausblick}\label{ch:zusammenfassung-und-ausblick}


\section{Zusammenfassung}\label{sec:zusammenfassung}
Es wurde eine Software angefertigt, welche alle Vorgaben bezüglich der geforderten Funktionalität erfüllt.
Der gesamte Ablauf, vom Einlesen eines externen Wörterbuchs, über das eigentliche Mapping der Tasten und der Datenstruktur, bis hin zur Behandlung von Fehlerfällen und das Schreiben eines neuen Wörterbuchs, wurde korrekt abgebildet.\\

Mithilfe dieser Entwicklung können vereinfacht Texteingaben auf Tastaturen mit gleichem Aufbau, wie die eines Mobiltelefons, getätigt werden.\\
Bei der Implementierung wurde stets darauf geachtet, die Modularisierung einzuhalten, um mögliche Erweiterungen einfach einbinden zu können.
Eine ausführliche Test- und Entwicklungsdokumentation und ein Programmablaufplan geben projektfremden Personen, besonders Entwicklern einen tiefen Einblick in die Software und ihrer Funktionsweise.
Weitergehend wurden sprechende Variablennamen mit Namenskonventionen gewählt, um die Wartbarkeit des Codes maximal zu halten.


\section{Ausblick}\label{sec:ausblick}
Die entwickelte Software kann vielseitig erweitert und verbessert werden.

Eine mögliche Verbesserung wäre eine striktere Kontrolle der externen Wörterbücher.
Aktuell werden die Dateien nur minimal auf syntaktische Anforderungen überprüft, der Inhalt der einzelnen Zeilen wird aber nicht validiert.

Des Weiteren könnte die größe des internen Wörterbuchs dynamische zur Laufzeit erweitert werden.
Im Moment gibt es eine feste maximale Größe, sodass das Wörterbuch irgendwann voll ist.
Die Tabellenstrukturen liegen dafür schon in der passenden Form vor.

Auch Programmausgaben im Benutzerdialog haben Verbesserungspotential.
Alle eingegebenen Sätze könnten zum Beispiel als Paragraf am Programmende ausgegeben werden.

Zudem könnte ein noch größerer Fokus auf COBOL-Code-Konventionen gelegt werden.
