\chapter{Benutzeranleitung}\label{ch:benutzeranleitung}


\section{Vorbereiten des Systems}\label{sec:vorbereiten-des-systems}

\subsection{Systemvoraussetzungen}\label{subsec:systemvoraussetzungen}
Um das Programm zu benutzen ist ein Windows- oder Linux-System vorausgesetzt.
Unter Linux muss zusätzlich \textit{Wine}~installiert werden.

\subsection{Installation}\label{subsec:installation}
Die Installation des Programms erfolgt über das Entpacken der~.zip-Datei.
Die ausführbare .exe befindet sich im Anschluss im Unterordner \textit{bin}.
Außerdem ist es notwendig, dass unter Windows die PATH-Variable den Pfad zu GnuCOBOL/bin enthält.

\section{Programmaufruf}\label{sec:programmaufruf}
Um das Programm zu starten, muss die bin/TexteingabeHandy.exe aufgerufen werden.
Es öffnet sich ein Dialogfenster, mit welchem der Nutzer anschließend interagieren kann.


\section{Testen der Beispiele}\label{sec:testen-der-beispiele}
Die Beispiele können alle mittels der~.cmd-Dateien im bin Verzeichnis getestet werden.
Die TestAll.cmd enthält dabei alle Testfälle, welche automatisch nacheinander ausgeführt werden.
Hilfreich ist es, die ScreenBuffer-Size der Eingabeaufforderung auf eine höhere Zahl zu setzen, damit alle Zeilen in dem Dialogfenster bestehen bleiben.
Zu empfehlen ist hierbei eine Zahl über 1000.