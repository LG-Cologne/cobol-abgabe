\chapter{Benutzeranleitung}\label{ch:benutzeranleitung}


\section{Vorbereiten des Systems}\label{sec:vorbereiten-des-systems}

\subsection{Systemvoraussetzungen}\label{subsec:systemvoraussetzungen}
Um das Programm zu benutzen ist ein Windows-System vorausgesetzt.

\subsection{Installation}\label{subsec:installation}
Die Installation des Programms erfolgt über das Entpacken der~.zip-Datei.
Die ausführbare .exe befindet sich im Anschluss im Unterordner \textit{bin}.
Außerdem ist es notwendig, dass unter Windows die PATH-Variable den Pfad zu GnuCOBOL/bin enthält.

\section{Programmaufruf}\label{sec:programmaufruf}
Um das Programm zu starten, muss die bin/Gruppenwechsel.exe aufgerufen werden.

\section{Testen der Beispiele}\label{sec:testen-der-beispiele}
Die Beispiele sind im Ordner \textit{files/examples} zu finden. Um diese zu testen, müssen diese händisch in den Ordner \textit{files} abgelegt und in \textit{JOURNAL.txt} umbenannt werden. Anschließend kann das Programm gestartet werden.
Alternativ ist zu empfehlen, die eigentliche \textit{JOURNAL.txt} mit den Inhalt des zu testenden Beispiel zu überschreiben.