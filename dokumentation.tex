\documentclass[
    12pt, % Schriftgröße
    oneside, % zweiseitiger Modus
    ngerman, % deutsches Dokument
    BCOR=0mm, % Bindungskorrektur
    DIV=10 % Division (Anzahl Spalten/Zeilen pro Seite, bestimmt implizit Margins)
]{scrreprt}

\newcommand{\titleDocument}{Abgabeübung COBOL Dreiecksberechnung}
\newcommand{\authorDocument}{Leon Jarosch}
\newcommand{\subjectDocument}{Abgabeübung}
\newcommand{\locationDocument}{Köln}
\newcommand{\dateDocument}{\today} % Alternativ z.B. 30.~September 2021

\title{\titleDocument}
\author{\authorDocument}
\date{\dateDocument}

\usepackage{dokumentation}

\begin{document}
    % ============ Anfang =============
    % Titelseite
    \include{title}

    % Eidesstattliche Erklärung und Abstrakt
%    \begingroup
    % keine Seitenzahl und kein running header
%    \thispagestyle{empty}
%    \renewcommand*{\chapterpagestyle}{empty}

%    \cleardoubleoddpage % Abstrakt rechts

%    \glsresetall % alle bereits genutzten Akronyme wieder zurücksetzen
%    \endgroup

    % Inhaltsverzeichnis
%    \cleardoubleoddpage % Inhaltsverzeichnis rechts
%    \thispagestyle{plain}
    \tableofcontents

    % =========== Zahlenteil ===========
    \chapter{Programmbeschreibung}\label{ch:programmbeschreibung}


\begin{figure}
    \centering
    \includegraphics[width=\linewidth]{images/Programmablauf.jpg}
    \caption{Programmablauf}
\end{figure}



\begin{figure}
    \centering
    \includegraphics[width=\linewidth]{images/DETERMINE RECTANGULARITY.jpg}
    \caption{Winkelart}
\end{figure}

\begin{figure}
    \centering
    \includegraphics[width=\linewidth]{images/DETERMINE-TRIANGLE-TYPE.jpg}
    \caption{Dreiecksart}
\end{figure}

\begin{figure}
    \centering
    \includegraphics[width=\linewidth]{images/CALC-PERIMITER.jpg}
    \caption{Umfang}
\end{figure}

\begin{figure}
    \centering
    \includegraphics[width=\linewidth]{images/CALC-PERIMITER.jpg}
    \caption{Oberfläche}
\end{figure}

\begin{figure}
    \centering
    \includegraphics[width=\linewidth]{images/Prepare Output.jpg}
    \caption{Ausgabevorbereitung}
\end{figure}

\begin{figure}
    \centering
    \includegraphics[width=\linewidth]{images/display_output.jpg}
    \caption{Ausgabe}
\end{figure}

    \chapter{Verfahrensbeschreibung}\label{ch:verfahrensbeschreibung}


\section{Mathematischer Hintergrund}\label{sec:mathematischer_hintergrund}
Das System arbeitet mit verschiedenen mathematischen Verfahren mit welchen die benötigten Berechnungen durchgeführt werden.

\subsection{Formel von Heron}\label{subsec:formel_von_heron}
Zum berechnen des Flächeninhalts eines Dreiecks wird die Formel von Heron verwendet.
\\
Der Satz von Heron besagt, dass die Fläche eines Dreiecks durch die Länge seiner Seiten berechnet werden kann. Mathematisch ausgedrückt:

\begin{align}
    A=\sqrt{s(s-a)(s-b)(s-c)}\\
\end{align}
Wobei $s$ für die Hälfte des Umfangs steht:
\begin{align}
    s=\frac{a+b+c}{2}
\end{align}

\pagebreak
\subsection{Satz des Pythagoras}\label{subsec:satz_des_pythagoras}
Zum überprüfen ob ein Dreieck rechtwinklig ist, wird der Satz des Pythagoras verwendet.
\\
Der Satz des Pythagoras besagt, dass in einem rechtwinkligen Dreieck die Summe der Kathetenquadrate gleich dem Hypothenusenquadrat ist. Mathematisch ausgedrückt:
\begin{align}
    a^2+b^2=c^2\\
\end{align}
\begin{figure}
    \centering
    \includegraphics[width=0.5\linewidth]{images/pythagoras.png}
    \caption{Satz des Pythagoras}
    \label{fig:pythagoras}
\end{figure}

    \chapter{Testdokumentation}\label{ch:testdokumentation}
Im folgenden Testfälle mit welchem das Programm getestet wurde.

\definecolor{fhfarbe}{HTML}{00605E}


\section{Vordefinierte Tests}\label{sec:definierte_tests}


% \begin{tabular}{|c|c|c|c|c|c|}
%     \hline
%     a & b & c & U & F & Art \\
%     \hline
%     5 & 3 & 4 & 12 & 6,00 & r, s \\
%     \hline
%     11 & 11 & 10 & 32 & 48,990 & nr, gsch \\
%     \hline
%     29 & 29 & 29 & 87 & 364,164 & nr, gs \\

%     \hline
% \end{tabular}

\begin{figure}
    \centering
    \includegraphics[width=\linewidth]{images/Testfall1.png}
    \caption{Testfall 1}
\end{figure}

\begin{figure}
    \centering
    \includegraphics[width=\linewidth]{images/Testfall2.png}
    \caption{Testfall 2}
\end{figure}

\begin{figure}
    \centering
    \includegraphics[width=\linewidth]{images/Testfall3.png}
    \caption{Testfall 3}
\end{figure}

\section{Ergänzende Tests}\label{sec:ergänzende_tests}

\begin{figure}
    \centering
    \includegraphics[width=\linewidth]{images/Testfall4.png}
    \caption{Testfall 4}
\end{figure}

\begin{figure}
    \centering
    \includegraphics[width=\linewidth]{images/Testfall5.png}
    \caption{Testfall 5}
\end{figure}

\begin{figure}
    \centering
    \includegraphics[width=\linewidth]{images/Testfall6.png}
    \caption{Testfall 6}
\end{figure}
\begin{figure}

    \centering
    \includegraphics[width=\linewidth]{images/Testfall7.png}
    \caption{Testfall 7}
\end{figure}
    \addtocontents{toc}{\protect\newpage}
    % ============= Buchstabenteil ==============
    \renewcommand{\thechapter}{\Alph{chapter}}%
    \setcounter{chapter}{0}
    % \chapter{Entwicklungsumgebung}\label{ch:entwicklungsumgebung}

Das Programm wurde mithilfe der OpenCobolIDE (\url{https://launchpad.net/cobcide/+download}) in der Version 4.7.6 geschrieben.
Dabei handelt es sich um eine leichtgewichtige COBOL Entwicklungsumgebung, die als Compiler GnuCOBOL 2.0.0 (\url{https://sourceforge.net/projects/gnucobol/}) verwendet.
Der GnuCOBOL-Compiler übersetzt den COBOL-Source-Code in ein C-Programm und erzeugt daraus, mit einem nativen C-Compiler, eine ausführbare Datei.

Im Entwicklungsprozess wurde zur Versionsverwaltung GitHub (\url{https://github.com/}) verwendet.
Die dort angebotenen Remote-Repositories ermöglichen eine effiziente Zusammenarbeit im Team.

Alle Entwicklungsschritte wurden auf Systemen mit Windows 10 Betriebssystem (\url{https://www.microsoft.com/de-de/software-download/windows10}) durchgeführt.

\begin{figure}[htb]
    \centering
    \begin{minipage}{.5\textwidth}
        \centering
        \includegraphics[width=.4\linewidth]{images/opencobol-logo}
    \end{minipage}%
    \begin{minipage}{.5\textwidth}
        \centering
        \includegraphics[width=.4\linewidth]{images/Gnu-COBOL}
    \end{minipage}
    \caption{Logos von OpenCobolIDE und GnuCOBOL.}
    \label{fig:cobol-logos}
\end{figure}

\begin{figure}[htb]
    \centering
    \includegraphics[width=.15\linewidth]{images/GitHub-Logo}
    \caption{
        GitHub-Logo.
    }
    \label{fig:github}
\end{figure}
 zusammen legen mti Hilfmittel
    \chapter{Verwendete Hilfsmittel}\label{ch:verwendete-hilfsmittel}

Als Hilfsmittel wurden hauptsächlich die Inhalte der, von Prof. Dr. rer. nat. Karola Merkel (\url{https://www.fh-aachen.de/fachbereiche/medizintechnik-und-technomathematik/einrichtungen/sp-studienort-koeln/kontakt}) angebotenen, Vorlesung \glqq COBOL\grqq{} verwendet.
Ergänzend dazu wurde die offizielle COBOL-Dokumentation von IBM (\url{https://www.ibm.com/docs/en}) zurate gezogen.

Zudem konnten unterschiedliche Fragen durch das Durchsuchen von Foren gelöst werden.
Besonders häufig konnten das \glqq Expertforum\grqq{} (\url{https://ibmmainframes.com/forum-1.html}) und \glqq stackoverflow\grqq{} (\url{stackoverflow.com}) Antworten liefern.
    \chapter{Erklärung}\label{ch:erklaerung}

Hiermit versichere ich, dass ich die vorliegende Arbeit mit dem Thema
\begin{quote}
    \textit{\titleDocument}
\end{quote}
selbstständig verfasst und keine anderen als die angegebenen Quellen und Hilfsmittel benutzt habe.
Alle Ausführungen, die anderen Schriften wörtlich oder sinngemäß entnommen wurden, sind kenntlich gemacht und die Arbeit ist in gleicher oder ähnlicher Fassung noch nicht Bestandteil einer Studien- oder Prüfungsleistung.
\\
\\
Mein Beitrag zur Abgabe:
\begin{itemize}[noitemsep]
    \item Ich habe große Teile des Benutzerdialogs implementiert.\\
    COBOL-Paragrafen: \glqq Benutzer-Dialog\grqq{},  \glqq Ermittle-Wort\grqq{},  \glqq Wort-Auswahl\grqq{},  \glqq Finde-Moeglichkeiten\grqq{}, \glqq Sortiere-Nach-Haeuf\grqq{}, \glqq Explizit-Eingabe\grqq{}, \glqq Suche-Wort-In-WBuch\grqq{} und \glqq Konstruiere-Wort\grqq{}
    \item Kapitel der Dokumentation: ~\ref{sec:fehlerarten}, ~\ref{sec:fehlerbehandlung}, ~\ref{subsec:datenstrukt}, ~\ref{sec:definierte-tests}, ~\ref{sec:explizitmodus-tests}, ~\ref{sec:semantik-explizitmodus-tests},
    ~\ref{sec:wortvorschläge-tests}, ~\ref{sec:sonderfall-tests}, ~\ref{ch:entwicklungsumgebung}, ~\ref{ch:verwendete-hilfsmittel}
    \item automatische Tests: TestExplizitEingabe.cmd, TestFunktionalitaet.cmd, TestReihenfolgeWortvorschlag.cmd, TestSemantikExplizitEingabe.cmd und TestSonderfall.cmd
\end{itemize}
\vspace*{2cm}

\begingroup
\setlength{\parindent}{0pt} % keine Einrückung bei neuen Absätzen in diesem Bereich

\locationDocument, den \dateDocument
\bigskip
\bigskip

% gewünschte Breite der Unterschriftslinie
\newlength{\widthbox}
\settowidth{\widthbox}{\locationDocument, den \dateDocument}

\makebox[\widthbox]{\hrulefill}\\
Ben Pietsch
\endgroup
\newpage

Hiermit versichere ich, dass ich die vorliegende Arbeit mit dem Thema
\begin{quote}
    \textit{\titleDocument}
\end{quote}
selbstständig verfasst und keine anderen als die angegebenen Quellen und Hilfsmittel benutzt habe.
Alle Ausführungen, die anderen Schriften wörtlich oder sinngemäß entnommen wurden, sind kenntlich gemacht und die Arbeit ist in gleicher oder ähnlicher Fassung noch nicht Bestandteil einer Studien- oder Prüfungsleistung.
\\
\\
Mein Beitrag zur Abgabe:
\begin{itemize}[noitemsep]
    \item Validierung der Benutzereingaben:\\
    COBOL-Paragrafen: \glqq Pruefe-Eingabe\grqq{} und \glqq Pruefe-Explizit-Eingabe\grqq{}
    \item Kapitel der Dokumentation: ~\ref{ch:programmbeschreibung}, ~\ref{ch:zusammenfassung-und-ausblick},
    \item automatische Tests: TestT9-Eingabe.cmd
\end{itemize}
\vspace*{2cm}

\begingroup
\setlength{\parindent}{0pt} % keine Einrückung bei neuen Absätzen in diesem Bereich

\locationDocument, den \dateDocument
\bigskip
\bigskip

% gewünschte Breite der Unterschriftslinie
%\newlength{\widthbox}
\settowidth{\widthbox}{\locationDocument, den \dateDocument}

\makebox[\widthbox]{\hrulefill}\\
Natalie Fritzen
\endgroup
\newpage

Hiermit versichere ich, dass ich die vorliegende Arbeit mit dem Thema
\begin{quote}
    \textit{\titleDocument}
\end{quote}
selbstständig verfasst und keine anderen als die angegebenen Quellen und Hilfsmittel benutzt habe.
Alle Ausführungen, die anderen Schriften wörtlich oder sinngemäß entnommen wurden, sind kenntlich gemacht und die Arbeit ist in gleicher oder ähnlicher Fassung noch nicht Bestandteil einer Studien- oder Prüfungsleistung.
\\
\\
Mein Beitrag zur Abgabe:
\begin{itemize}[noitemsep]
    \item Einlesen von externen Wörterbüchern und das sortieren und schreiben des internen Wörterbuchs.\\
    COBOL-Paragrafen: \glqq Auswahl-WBuch\grqq{},  \glqq Einlesen-WBuch.\grqq{},  \glqq Lese-Satz\grqq{} und \glqq Schreibe-WBuch-Sortiert\grqq{}
    \item Kapitel der Dokumentation: ~\ref{sec:interpretation-der-aufgabe}, ~\ref{sec:anforderung-an-das-programm}, ~\ref{subsec:eingabe}, ~\ref{subsec:verarbeitung}, ~\ref{subsec:ausgabe}, ~\ref{sec:wörterbücher-tests}, ~\ref{ch:benutzeranleitung}
    \item automatische Tests: TestAlle.cmd, und TestWBuchEinlesen.cmd
\end{itemize}
\vspace*{2cm}

\begingroup
\setlength{\parindent}{0pt} % keine Einrückung bei neuen Absätzen in diesem Bereich

\locationDocument, den \dateDocument
\bigskip
\bigskip

% gewünschte Breite der Unterschriftslinie
%\newlength{\widthbox}
\settowidth{\widthbox}{\locationDocument, den \dateDocument}

\makebox[\widthbox]{\hrulefill}\\
Leonhard Keßler
\endgroup
    \chapter{Aufgabenstellung}\label{ch:aufgabenstellung}

\includepdf[pages=-]{images/ABGABEUEBUNG 2023_23.pdf}
    \include{chapters/G-Quellcode}
\end{document}
